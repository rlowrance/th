% thesis-prologue.tex
% latex document class, packages, local commands

% ENTERING LATEX'S SPECIAL CHARACTERS: # $ % & ~ _ ^ \ { } @ " |
%  \#             for #
%  \$             for $
%  \%             for %
%  \&             for & (ampersand)
%  \~{}           for ~ (tilde)
%  \_             for _ (underscore)
%  \^{}           for ^
%  \textbackslash for \
%  \{             for {
%  \}             for }
%  @              for @
%  `` or ''       for "
%  \textbar       for | in text mode
% \mid            for | in math mode (insert space before and after)

% OTHER ENTRIES
%  \ldots for ... (ellipsis)

% File outline and command nears the top are based on 
% George Gratzer, Math Into Latex, Third Edition, Birkhausser
% Boston, 2004.

\documentclass[10pt]{amsbook}
%\setlength{\textwidth}{5in}

\usepackage{amssymb,latexsym,amsmath}
\usepackage{listings}
%\usepackage[all]{xy}   %xy-pic

% graphics
\usepackage{graphicx}    % needed for \includegraphics
% Tell \includegraphics where to search for files
% see Gratzer p 317
% The trailing / is required
\graphicspath{%
{/home/roy/Dropbox/nyu-real-estate/repp-repo.git/src/local-weighted-regression-2/src/}
}
\usepackage{epstopdf}   % allow eps files as graphics input
\usepackage{caption}    % allow line breaks \\ in captions

% don't indent first list of a paragraph
%\setlength{\parindent}{0pt}   
% increase spacing between paragraphs
\setlength{\parskip}{1ex plus 0.5ex minus 0.2 ex}
% don't align right margin
\raggedright

% proclamations
%\newtheorem{corollary}{Corollary}
%\newtheorem{definition}{Definition}
\newtheorem{lemma}{Lemma}
%\newtheorem{notation}{Notation}
%\newtheorem{proposition}{Proposition}
%\newtheorem{theorem]{Theorem}

%%%%%% commands local to this document
\newcommand{\code}[1]{\texttt{#1}}
\newcommand{\term}[1]{\emph{#1}}
\newcommand{\blanks}{\_\_\_}
\newcommand{\blank}{\textunderscore\textunderscore\textunderscore}

\DeclareMathOperator*{\argmax}{arg\,max}
\DeclareMathOperator*{\argmin}{arg\,min}

% doublespace document; see Gratzer page 150
%\renewcommand{\baselinestretch}{2}

