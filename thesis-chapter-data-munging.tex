% vim: textwidth=72



% STYLE GUIDE
% not dataset, but data set
%% thesis-prologue.tex
% latex document class, packages, local commands

% ENTERING LATEX'S SPECIAL CHARACTERS: # $ % & ~ _ ^ \ { } @ " |
%  \#             for #
%  \$             for $
%  \%             for %
%  \&             for & (ampersand)
%  \~{}           for ~ (tilde)
%  \_             for _ (underscore)
%  \^{}           for ^
%  \textbackslash for \
%  \{             for {
%  \}             for }
%  @              for @
%  `` or ''       for "
%  \textbar       for | in text mode
% \mid            for | in math mode (insert space before and after)

% OTHER ENTRIES
%  \ldots for ... (ellipsis)

% File outline and command nears the top are based on 
% George Gratzer, Math Into Latex, Third Edition, Birkhausser
% Boston, 2004.

\documentclass[10pt]{amsbook}
%\setlength{\textwidth}{5in}

\usepackage{amssymb,latexsym,amsmath}
\usepackage{listings}
%\usepackage[all]{xy}   %xy-pic

% graphics
\usepackage{graphicx}    % needed for \includegraphics
% Tell \includegraphics where to search for files
% see Gratzer p 317
% The trailing / is required
\graphicspath{%
{/home/roy/Dropbox/nyu-real-estate/repp-repo.git/src/local-weighted-regression-2/src/}
}
\usepackage{epstopdf}   % allow eps files as graphics input
\usepackage{caption}    % allow line breaks \\ in captions

% don't indent first list of a paragraph
%\setlength{\parindent}{0pt}   
% increase spacing between paragraphs
\setlength{\parskip}{1ex plus 0.5ex minus 0.2 ex}
% don't align right margin
\raggedright

% proclamations
%\newtheorem{corollary}{Corollary}
%\newtheorem{definition}{Definition}
\newtheorem{lemma}{Lemma}
%\newtheorem{notation}{Notation}
%\newtheorem{proposition}{Proposition}
%\newtheorem{theorem]{Theorem}

%%%%%% commands local to this document
\newcommand{\code}[1]{\texttt{#1}}
\newcommand{\term}[1]{\emph{#1}}
\newcommand{\blanks}{\_\_\_}
\newcommand{\blank}{\textunderscore\textunderscore\textunderscore}

\DeclareMathOperator*{\argmax}{arg\,max}
\DeclareMathOperator*{\argmin}{arg\,min}

% doublespace document; see Gratzer page 150
%\renewcommand{\baselinestretch}{2}



\documentclass[10pt]{amsart}
%\documentclass[10pt]{amsbook}
%\setlength{\textwidth}{5in}

\usepackage{amssymb,latexsym,amsmath}
\usepackage{listings}
%\usepackage[all]{xy}   %xy-pic

% graphics
\usepackage{graphicx}    % needed for \includegraphics
% Tell \includegraphics where to search for files
% see Gratzer p 317
% The trailing / is required
\graphicspath{%
{/home/roy/Dropbox/nyu-real-estate/repp-repo.git/src/local-weighted-regression-2/src/}
}
\usepackage{epstopdf}   % allow eps files as graphics input
\usepackage{caption}    % allow line breaks \\ in captions

% don't indent first list of a paragraph
%\setlength{\parindent}{0pt}   
% increase spacing between paragraphs
\setlength{\parskip}{1ex plus 0.5ex minus 0.2 ex}
% don't align right margin
\raggedright

% proclamations
%\newtheorem{corollary}{Corollary}
%\newtheorem{definition}{Definition}
\newtheorem{lemma}{Lemma}
%\newtheorem{notation}{Notation}
%\newtheorem{proposition}{Proposition}
%\newtheorem{theorem]{Theorem}

%%%%%% commands local to this document
\newcommand{\code}[1]{\texttt{#1}}
\newcommand{\term}[1]{\emph{#1}}
\newcommand{\blanks}{\_\_\_}
\newcommand{\blank}{\textunderscore\textunderscore\textunderscore}

\DeclareMathOperator*{\argmax}{arg\,max}
\DeclareMathOperator*{\argmin}{arg\,min}
\begin{document}
\title{Input Processing}
\author{Roy E. Lowrance}
\date{\today}
\email{roy.lowrance@gmail.com}
\maketitle

%\SweaveOpts{concordance=TRUE}
<<control, include=FALSE>>=
options(warn = 2)  # turn warnings into errors
source('DirectoryWorking.R')
Control <- function() {
    working <- DirectoryWorking()
    control <-
        list( path.in.transactions = paste0(working, 'transactions.RData')
             ,path.in.subset = paste0(working, 'transactions-subset1.RData')
             ,path.in.deeds = paste0(working, 'deeds-al-g.RData')
             ,path.in.parcels = paste0(working, 'parcels-sfr.RData')
             ,cache = FALSE
             #,cache = TRUE
             #,include = FALSE
             ,include = TRUE
        )
    control
}

control <- Control()
str(control)
opts_chunk$set(cache = control$cache)
opts_chunk$set(include = control$include)
@

<<Commas>>=
Commas <- function(i) {
    # insert commas into an integer
    format(i, big.mark = ',')
}
# test
Commas(123456)
@

<<About>>=
About <- function(x) {
  # return rounded number like 1,200,000 (without the commas)
  million <- 1000000
  million * round(x / million, 1)
}
@

<<deeds>>=
Deeds <- function(control) {
    deeds.loaded <- load(control$path.in.deeds)
    str(deeds.loaded)
    list( info = count
          ,df = deeds.al.grant
          ,control = control
    )
}
deeds <- Deeds(control)
str(deeds$info)
str(deeds$control)
@

<<parcels>>=
Parcels <- function(control) {
    parcels.loaded <- load(control$path.in.parcels)
    str(parcels.loaded)
    list( info = info
         ,df = parcels.sfr
         ,control = control
         )
}
parcels <- Parcels(control)
str(parcels$info)
str(parcels$control)
@

<<subset>>=
Subset <- function(control) {
    subset.loaded <- load(control$path.in.subset)
    str(subset.loaded)
    list( info = info
         ,df = transactions.subset1
         ,control = control
    )
}
subset <- Subset(control)
str(subset$info)
str(subset$control)
subset.nrow <- nrow(subset$df)
subset.nrow
@

<<transactions>>=
Transactions <- function(control) {
    transactions.loaded <- load(control$path.in.transactions)
    str(transactions.loaded)
    list( info = info
         ,df = transactions
         ,control = control
    )
}
# a bug in Knitr prevents something from reloaded
# Maybe that something is the dataframe
transactions <- Transactions(control)
str(transactions$info)
str(transactions$control)
@


We start with real estate data for Los Angeles Country: the tax roll for
2008, containing about
\Sexpr{Commas(About(parcels$info$nrow.all))} 
parcels, and 25 years of deeds, about
\Sexpr{Commas(About(deeds$info$num.all.deeds))} 
observations,
ending early in 2009. These data came from
CoreLogic. We supplemented them with data from the U.S. Census Bureau
from the year 2000 and, from a geocoding service, the latitude and
longitude of many of the parcels. We join a subset of these files to create a
transactions file for single-family residences that were sold at arms-length. Here
a transaction is a parcel and the price of the parcel on a date that
it sold. We have about
\Sexpr{Commas(About(transactions$info$nrow.transactions))}
transactions. The transactions file contains observations with unusual values which
we presume to be erroneous. For example, there are houses with zero rooms.
So we create a subset (``subset1'') containing only
transaction observations with values we define to be reasonable. Most of the analysis
uses subset1, which contains about
\Sexpr{Commas(About(subset$info$num.with.possible.duplicates))}
observations.

This chapter provides the details on how this was done. It contains these
sections:
\begin{itemize}
\item A description of the input files
\item How the input files were joined into the transactions file.
\item How a subset of the transactions was selected.
\item An optimization designed to speed up testing of programs using the
subset.
\end{itemize}

\section{Input Files}

The tax roll is used by the tax assessor to prepare and send property tax
bills. In Los Angeles Country, the initial real estate bills are send
starting November 1. The tax roll used in this work is as of November 1,
2008, representing tax dues in late 2008 and in 2009.

The eight tax roll files we use are assembled by CoreLogic, which obtains the
original files for Los Angeles County, cleans them up, augments them with other data, and sells
them. There is one record (CoreLogic type 2580) for every parcel.

The fields in each record are in these groups:

\begin{itemize}

\item The parcel identifier, called the Assessor Parcel Number (APN).
This value is presented twice, once formatted with hyphens and once as a
plain number field.  The number field is not always numeric and does not
always have the correct number of digits, so the two fields are analyzed
to infer the ``best'' APN.

\item Information on the parcel itself: it's census tract, its latitude
and longitude (though these fields are not populated in our data, so
that we have used a separate geocoding file), location on maps, the
universal land use code (LUSEI), and so forth. The LUSEI field is used
to identify whether the parcel is for a single-family residence.

\item Information on the subdivision, primarily its location in reference books.

\item The address of the property including its 9-digit zip code.

\item Information on the owner, which is not populated.

\item A series of fields describing the assessment for the parcel.
  California Propositions 7 and 13 force the assessment to not be an
  unbiased estimate of the market value of the property. Proposition 13
  constrains the assessment to not grow beyond the last sale price of
  the property by more than two percent annually, thus biasing
  assessments to be on average lower than property values in periods of
  high real estate inflation. Proposition 7 reduces the assessment to an
  estimated market value upon successful petition of the property owner,
  thus biasing the assessment to be on average higher than property
  values, because not all owners will petition to have their assessments
  reduced when the real estate market is falling.

\item Information on the most recent sale of the property. We don't use this
information and rely instead on the information in the deeds. The two
sources do not always agree.

\item Information on the mortgage. We don't use any mortgage information in
this work.

\item Information on the prior sale. We again rely on the deeds for this type
of information.

\item A description of the lot including its size in acres and square feet.

\item A description of the primary building, including the year built, number 
of rooms, number of bedrooms, number of bathrooms, whether it has a swimming
pool. Many of the values are missing, so we use a subset of the values that
are often present.

\item The legal description of the property. We don't use this.

\end{itemize}

A deed transfers ownership or legal rights to a property. When a
property is sold or encumbered, one files an appropriate deed with the
registrar of deeds. The deeds used in this work are for the 25 years
ending late in 2009.

The eight deeds files we use are assembled by CoreLogic, which obtains
the original files, cleans them up, augments them with other data, and
sells them. There is one record (CoreLogic type 1080) for every deed.

The fields in each record are in these groups:

\begin{itemize}

\item The parcel identifier, the APN. This is coded as in the tax roll file and
has similar issues.

\item A description of the owner. We don't use this.

\item The owner's mailing address. We don't use this.

\item Property information. We rely on the tax roll for this type of information
and hence do not use these fields.

\item Information on the sale, including the sale date, the price, how
  many APNs were in the transaction, the type of deed, and the primary
  category code (PRICATCODE) reporting whether the transaction was at
  arms-length. We use only arms-length transactions in this work. We use
  only grant deeds (these are deeds of sale) and trust deeds (sales
  supported by a mortgage) in this work.

\end{itemize}

The census file was downloaded from the U.S. Census. It is for the year 2000 and
contains records for census tracts in Los Angeles Country.

The geocoding file was produced by GeoLytics, Inc. It contains latitudes and
longitudes for many of the parcels in Los Angeles.

\section{Creating the Transactions File}

We join the input files to create a transactions file which we then
subset to form the main data file for most of the analysis. This section 
describes the steps.

The major steps are these:
\begin{itemize}
  \item Select just the arms-length sale deeds.
  \item Select just the parcels containing single-family-residences.
  \item Create additional features for the zip codes and census tracts.
  \item Create additional features for the census tracts.
  \item Join all the files together.
  \item Pick a subset that has ``reasonable'' values.
  \item Split the subset into individual features.
\end{itemize}

Details of each of these steps are in the subsections and sections that follow.

\subsection{Select arms-length sale deeds}

The deeds file classifies every record as to whether it was an
arms-length sale or not.  How a deed is classified as arms-length is not known for sure, but
presumedly relies on the type of deed and the relationship if any
between the seller and buyer.

Our work uses only deeds classified as arms-length sales as recorded
in the PRICATCODE (primary category code) field. The deeds files from CoreLogic contain
\Sexpr{Commas(deeds$info$num.all.deeds)}
deeds of which  
\Sexpr{Commas(deeds$info$num.is.arms.length)} 
are classified as arms-length.

The document type code field records the type of the deed. The deeds of
interest are those that transfer ownership of a property. There are the
grant deeds. There are 
\Sexpr{Commas(deeds$info$num.is.grant.deed)} 
grant deeds in the deeds files.

We are interested in deeds that are both arms-length and grant deeds.
There are 
\Sexpr{Commas(deeds$info$num.is.arms.length.and.grant.deed)} 
such deeds.

\subsection{Select single-family residences}

The tax assessor classifies each parcel according to its primary use.
One of the uses is as a single-family residence.

Our work uses only parcels classified as single-family residences as
recorded in the LUSEI field. The tax roll files from CoreLogic contain
\Sexpr{Commas(parcels$info$nrow.all)} 
tax roll records of which 
\Sexpr{Commas(parcels$info$nrow.sfr)}
are classified as single-family
residences.

\subsection{Create additional features for zip codes and census tracts}

A potentially-informative feature of a parcel is whether it is near 
industry, a park, shopping, or a school. Perhaps the first of these
characteristics detracts from attractiveness and the others increase
attractiveness.

These features are not directly reflected in the tax roll file and hence
must be deduced. We have latitudes and longitudes only for residences,
not for parcels containing industry, parks, retailers, or schools. Hence
we resort to determining whether 5-digit zip codes and census tracts
contain industry, parks, retailers, and schools.

Thus we have two additional input files that must be joined: one stating
whether every 5-digit zip code has any of the features of interest, the
other with the same information for every census tract.

\subsection{Create additional features for the census tracts}

The features we want to have for the census tracts are the average
commuting time (perhaps longer commutes lower property values), the
median household income (perhaps neighborhoods with higher incomes also
have higher property values), and the fraction of houses that are
owner-occupied (perhaps higher ownership is associated with higher
property values).

None of these features is provided directly in the Census Bureau files,
but all are straight-forward to compute from the information provided.

\subsection{Join all the files together}

The transactions file is created by joining each of the input files.
\begin{itemize}
  \item The file of all arms-length sale deeds containing 
    \Sexpr{Commas(transactions$info$nrow.deeds.al)} records.
  \item The file of all single-family-residence parcels containing 
    \Sexpr{Commas(transactions$info$nrow.parcels.sfr)}
    records. These parcels are naturally joined into the deeds using the
    best APN field from each file. The resulting joined file has
    \Sexpr{Commas(transactions$info$nrow.deeds.parcels)} records. The
    unique key of each record is the APN and the recording date for the
    deed. Other fields in the joined records are information from the deeds file
    including, when available, the date of the sale and the price, and
    information from the tax roll file including, when available, a
    description of the property (lot size, interior space, number of
    rooms, and so forth).
  \item The file containing 
    \Sexpr{Commas(transactions$info$nrow.census)} records derived from
    the census tract data. The information in these records is appended
    to the joined deeds-parcels file using the census tract fields to
    line up the records.  A file with
    \Sexpr{Commas(transactions$info$nrow.census.deeds.parcels)} records
    results.
  \item The file containing
    \Sexpr{Commas(transactions$info$nrow.geocoding)} records from the
    geocoding provider.  This file has the APN as its primary key and
    usually contains the latitude and longitude of the corresponding
    parcels. These location values are appended onto the corresponding
    records in the census-deeds-parcel file. The resulting merged file
    has
    \Sexpr{Commas(transactions$info$nrow.census.deeds.parcels.geocoding)}
    records.
  \item The file containing 
    \Sexpr{Commas(transactions$info$nrow.derived.zip5)} records recording 
    features of the 5-digit zip codes and 
    \Sexpr{Commas(transactions$info$nrow.derived.census.tract)} similar 
    records from census tracts. These information in these files is
    appended onto the census-deeds-geocoding-parcels file to create the
    transactions file.
\end{itemize}


The resulting transactions file contains 
\Sexpr{Commas(transactions$info$nrow.transactions)} 
records.


\section{Pick a subset with reasonable values}

We would be finished with the data munging except that the transactions
file contains observations with very unusable values. For example,
there are single-family residences with no rooms and with sales prices in
the hundreds of millions of dollars.

This project assumed that observations with unreasonable values were
not recorded properly and chose to discard those observations. An alternative
would be to identify the missing or mis-coded values and to impute their
values.

<<Remaining>>=
Remaining <- function(used) {
    Commas(subset$info$num.transactions.al.sfr - used)
}
@

These judgements were applied to reject records and form ``subset1,''
the subset of transactions actually used in the analysis.
\begin{itemize}
  \item Assessed value. Transaction arising from parcels with an
    assess value exceeding the maximum sales price were discarded. How
    the maximum sales price was determined is described just below under ``Sale amount.''
    There are \Sexpr{Remaining(subset$info$assessed.value$num.not.too.large)} such.
  \item Document type code. Only transactions arising from grant
    deeds were retained. Grant deeds are sales. There are many other 
    types of deeds. Records with other types
    of deeds were excluded. There were
    \Sexpr{Remaining(subset$info$DOCUMENT.TYPE.CODE$grant.deed)}
    such.
  \item Effective year built. Transactions arising from properties
    without an effective year built were discarded. The effective year
    built is the year of the last major remodeling or the year the property
    is built. There are \Sexpr{Remaining(subset$info$effective.year.built)} such.
  \item Geocoding. Transactions for which either the latitude or
    longitude were missing were rejected. There were 
    \Sexpr{Remaining(subset$info$geocoding)} such.
  \item One building. Transactions arising from parcels with more
    than one building were rejected because we have only descriptions
    for one building. There are \Sexpr{Remaining(subset$info$one.building)}
    such.
  \item One parcel. Transactions arising from deeds that reported
    more than one sold parcel were rejected because there is no way to
    apportion the price to the parcels. There are 
    \Sexpr{Remaining(subset$info$one.parcel$num.one.apn)} such.
  \item Sale amount. Some deeds report extremely large prices.
    Research in the Wall
    Street Journal suggests that the highest transaction price in Los
    Angeles through the end of 2009 is
    \Sexpr{Commas(as.integer(subset$control$max.sale.amount))} dollars, 
    so observations with prices
    higher than this were rejected. 
    There were \Sexpr{Remaining(subset$info$sale.amount)} such.
  \item Sale code. The price on the deed might not be for the full
    value of the parcel. We rejected transactions that did not say the
    sales price was for the full amount. There were 
    \Sexpr{Remaining(subset$info$sale.code$sale.price.full)} such.
  \item Sale date. None of the deeds are missing recording dates. Some are missing
  sale dates. When a
    sales date is missing, it is imputed from the recording date. This
    imputation uses the average delay between sales and recording (
    \Sexpr{round(subset$info$mean.days.between.sale.and.recording.date)}
    days) as the best estimate for the missing sales date.
  \item Total rooms. We rejected transactions for houses reported to
    have less than \Sexpr{Commas(subset$control$min.total.rooms)} rooms. 
    There were \Sexpr{Remaining(subset$info$total.rooms)} such.
  \item Transaction type code. We rejected transactions for parcels
    that were not either new construction nor resales. Other
    possibilities include time shares, construction loans, and
    refinancing. There were 
    \Sexpr{Remaining(subset$info$TRANSACTION.TYPE.CODE$resale + 
    subset$info$TRANSACTION.TYPE.CODE$new.construction)} such.
  \item  Units number. We have the description for only the primary
    unit on the parcel, so we rejected observations with more than one
    unit. There were \Sexpr{Remaining(subset$info$units.number)} such.
  \item Year built. Transactions arising from properties with a
    missing year built were discarded. There are \Sexpr{Remaining(subset$info$year.built)} such.
\end{itemize}

Some transactions were discarded because of extremely high values in one
or more features. Observations that exceeded the \Sexpr{subset$control$max.percentile}th percentile of
reported values were discarded. Features subject to this protocol were
these:
\begin{itemize}
  \item Land square footage. The square footage of the land. The
    largest value in the transactions file is 
    \Sexpr{Commas(subset$info$land.square.footage$max.before.selection)};
     the largest value in
    the subset of retained transactions is 
    \Sexpr{Commas(subset$info$land.square.footage$max.after.selection)}. 
    There are \Sexpr{Remaining(subset$info$land.square.footage$num.selected)}
    observations with values exceeding the highest allowed percentile.
  \item Living square feet. The square footage of the livable part
    of the  house. The largest value in the transactions file is 
    \Sexpr{Commas(subset$info$living.square.feet$max.before.selection)};
    the largest value in the subset of retained transactions is 
    \Sexpr{Commas(subset$info$living.square.feet$max.after.selection)}.
    There are \Sexpr{Remaining(subset$info$living.square.feet$num.selected)} 
    observations with values exceeding the highest allowed
    percentile.
  \item Universal building square feet. The square footage inside
    the house. The largest value in the transactions file is 
    \Sexpr{Commas(subset$info$building.square.feet$max.before.selection)}; the
    largest value in the subset of the retained transactions is 
    \Sexpr{Commas(subset$info$building.square.feet$max.after.selection)}.
    There are \Sexpr{Remaining(subset$info$building.square.feet$num.selected)}
    observations with values exceeding the highest allowed
    percentile.
\end{itemize}

The resulting subset file (``subset1'') contains 
\Sexpr{Commas(subset.nrow)} 
records.

\section{Split the subset into individual features}

This part of the project uses R as the programming language. The output
of each of the processing steps is a data frame that is stored in R's
internal binary serialized format. The idea is to make it quick to read
in the data. CSV files can be created later if they are needed.

After I starting working with subset 1, I found that reading the
binary serialized format takes several minutes, and that is too long to
wait. So I created one final processing step, which is to split the
subset 1 data frame into individual features and write the features as
1-column data frames in serialized files.  Often an experiment needs
only a hand full of features, and reading just the features needed and
assembling them into a data frame for analysis is often quicker than reading
all the features and discarding most of them.

Since models will need transformed versions of the features, I
pre-computed those as well. For the continuous features will all
positive values, I created centered versions and centered versions of
the log of the values. For continuous features that can be zero, instead
of the log I used 1 plus the log.


\end{document}
 
% vim:fdm=syntax
